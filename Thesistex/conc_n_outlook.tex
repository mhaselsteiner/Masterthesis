\chapter{Conclusion and Outlook}
\label{sec:con_n_out}
Although the overall performance of the established Gul-parametrization is in most cases better than of the newly introduced one, we have reason to believe, that with the general advancement of the model, the explicit Clark+ visibility forecast will be able to outmatch it in the future. On the long run the overall improvement of AROME and in particular prognostic aerosols and advances in data assimilation should provide a decrease of the uncertainties and approximations, which are mainly responsible for the rather low scores of the Clark+-parametrization.\\ \\
Unfortunately, including data assimilation, would have exceeded the scope of this Master's thesis and was therefore not done. Data assimilation mainly improves forecasts within the first few hours. Since the outputs used in the evaluation were taken the earliest after nine hours of forecast duration, we suspect that data assimilation would not have had any effects.
However, for short-range forecasts data assimilation reduces uncertainties and thus, can lead to different results for the skill of the visibility parametrization, because of the increased accuracy of relative humidity and hydrometeor concentrations. It seems likely, that data assimilation would have a major influence on how well a certain perturbation scheme performs too. Especially within the first two hours, the uncertainties of some quantities decrease, whereas of others, e.g. aerosol distribution, they remain the same. As a consequence the ranges of the perturbation patterns will need to be adapted, if data assimilation is used. Testing the effect of data assimilation on the performance in short-range forecasts should be the subject of future studies, to investigate, if the new visibility parametrization can profit from it and might be used in nowcasting systems (NWP within six hours from base time). \\ \\
The results confirm that the severest weakness of our set-up is that the used aerosol data is climatological. Thus, more exceptional events, e.g. sudden increase of Saharan dust over Europe and the like, are not treated in the model. Also changes in the distribution of sea salt, which heavily contributes to the total aerosol concentration in coastal areas and varies strongly, is neglected. Since the new parametrization
relies heavily on aerosols, it is thought to provide significant improvement in performance if a prognostic aerosols scheme would be used. This is further confirmed by the large increase in ensemble skill, when aerosols are included in the perturbation scheme, as shown in the case studies. Tests with aerosol data from a high resolution aerosol transport model were run and also showed promising results. These tests were carried out by a different implementation of the parametrization in a python program that used AROME output and the mentioned aerosol data as input. The aerosol transport model is based on the weather research and forecasting (WRF) model and has not been coupled with AROME so far. Coupling them could be a way to incorporate high quality aerosol forecasts in the model, but is challenging for the operational mode due to the associated computational costs.
For the operational mode the MESO-NH package, which is partially used in AROME and open-source, already includes a simple prognostic aerosol scheme that could be harmonized and calibrated for use in the AROME model in future studies.  \\ \\
A change that is expected to improve the parametrization itself is, to not only consider the forecast visual range at the current location, but also at the adjacent grid points. This would lower the discrepancy of forecast and observation, because the human observers also analyse all directions, when taking measurements. But because of  how the model domain is split up for parallelization purposes, we were not able to implement this in the current version. It might be worth considering implementing such a method as  part of the post-processing of the output data. Similar techniques are already applied to probabilistic precipitation forecasts \cite{theis_hense_damrath_2005}\\ \\
Concerning the perturbation scheme, we found that a local perturbation can be a good strategy to introduce a more realistic spread for model parts or parametrizations, when global perturbations cannot be applied to the needed extent. This is not limited to the forecast of visibility, but can be applied to other forecast variables as well.
Especially when the model parts of interest have a large number of different complex dependencies, it is in many cases not feasible to implement a global perturbation of the corresponding quantities. Then local perturbation can help to still be able to obtain an estimate of reliability and the probability distribution of the forecast variable. \\
Another advantage of a local perturbation scheme is that for diagnostic quantities like visibility it can also be used in post-processing, if the prediction is not contained in the model. The perturbation scheme could also be further improved by fine tuning the set-up and a better pattern generator. \\
As illustrated and discussed in Chapter \ref{sec:results}, the perturbation of specific humidity resulted in a decrease of overall skill. Hence, decreasing the perturbation range drastically or simply excluding specific humidity from the perturbation scheme, is expected to increase the skill of the totally perturbed ensemble. It should also lead to a stronger correlation of the totally perturbed ensemble and the perturbed aerosol and the perturbed parameter ensemble, which seem the most promising indicated by their high overall performance, especially in winter.\\ \\
The severe shortcomings of the pattern generator, discussed in Section \ref{sec:pattern gen}, are briefly revised:  Due to the fact that the namelist settings cannot be directly related to spatial scales, scaling the perturbations is a process of trial and error. Additionally, the cut-off of sampled values distorts the probability density functions. This could be resolved by re-sampling values that lie outside the range.
Solving these issues is challenging, but worth it, since any future experiments that include perturbations would profit from it.
