\chapter{Introduction} 

The humankind has always been exposed to nature, especially the weather.  Change of climate or extreme weather events have erased whole cultures from our planet, so no wonder that since the very early days of humanity, we observed our atmosphere, tried to find patterns and predict upcoming weather events.  Already in 340 BC Aristotle wrote the `Meteorologica', which can be seen as the very first documented structured approach to weather forecasting.
For a long time, meteorology was a purely empirical science, because there was no deeper understanding of the underlying physical processes. It was only at the beginning of the 20th century when \citeauthor{abbe1901physical} \cite{abbe1901physical} was the first one to draw the connection between large scale atmospheric processes and physics.
From there on we have come a very long way to where we are today. Because weather prediction is an overlap of physics, mathematics and, more recently, computer science, meteorology has evolved hand in hand with those fields.
Due to enhancements in above-mentioned disciplines, forecasts improved rapidly in the last 40 years: one day per decade. Meaning that today the accuracy of a prediction four days ahead is the same as the accuracy of a prediction three days ahead ten years ago \parencite{bauer2015quiet}. Nevertheless researchers in universities, private enterprises, governmental and military institutions are still trying to push it further. This is because the weather affects us in so many different ways, that having skilful predictions caters to different, sometimes even opposed, needs and interests. It can be a major contributor to sustain peace and stability for humanity as a whole, especially because of its role in agriculture and food production, and it can also be a valuable tool for the prevention of economic losses due to weather events and for the management of natural disasters \cite{webster2013meteorology}.\\ \\
The basis of modern weather forecast is to  mathematically describe the atmosphere as a turbulent fluid, using thermodynamics and fluid mechanics, and to solve the corresponding equations numerically \cite{coiffier2011fundamentals, batkai2016mathematical, phillips1960problem}.\\
The atmosphere is an intrinsically chaotic system and therefore it is very challenging to model and predict its behaviour \cite{lorenz1963deterministic}. It is  crucial to take the full range of possible initial conditions into account by modelling different possible scenarios of one case. This requires enormous computational power and sophisticated parallelization techniques \cite{batkai2016mathematical, leutbecher2008ensemble}. The improvement of computational facilities in recent years enhanced accuracy of the models, which caused Numerical Weather Prediction (NWP) to become the main area of research in meteorology and it is evolving rapidly. \\ \\
The spatial and temporal resolution is constantly getting higher and so the NWP models need to be adapted \cite{aladin1, theis_hense_damrath_2005}. With each improvement in the resolution more processes can be calculated explicitly, instead of using inaccurate empirically gained representations, and as a result, the contribution of model error is going down \cite{bauer2015quiet}. Besides the efforts of adapting models to a higher resolution the second area of focus of research in NWP is ensemble forecasting: to determine and evaluate the uncertainty of the forecast. Therefore, different scenarios, of the same prediction are simulated and the output of the whole ensemble is analysed. The statistics obtained from the ensemble provide insight about what scenario can be expected with the corresponding probability \cite{batkai2016mathematical}.
One of the main contributors to model uncertainty are parametrizations \cite{ollinaho2017towards, leutbecher2008ensemble}. 
Parametrizations are needed, because a large number of different complex physical processes leads to a single atmospheric state \cite{coiffier2011fundamentals}. Most modern NWP models use a temporal resolution of approximately 60 seconds. Due to limited computational capacities it is impossible to simulate all the underlying physical processes explicitly, for every time step for which the atmospheric state is determined. Hence, many of these simulations are replaced by mainly empirically gained representations, the parametrizations.
Paired with the uncertainties of the initial state and the chaotic nature of the atmosphere, the use of statistical methods becomes inevitable \cite{palmer2009stochastic, buizza1999stochastic, leutbecher2008ensemble}.\\
Depending on the purpose of the forecast, different variables of NWP models are of interest: Obviously the thermodynamic state variables, but often also additional ones e.g. accumulated precipitation for estimating the risk of flooding \cite{theis_hense_damrath_2005}, or sunshine duration to plan, when to carry out certain tasks in agriculture.\\
In the course of this Master's thesis, we are interested in predicting horizontal visibility, where especially for air traffic there is a demand for reliable forecasts.
Visibility is a measure for the shortest horizontal distance for which an object is still visible to a human observer, considering all directions \parencite{clark2008prediction, WMO, price2007advanced}.
It is effected by a great number of non-linear processes like precipitation and radiative processes.
Most parametrizations implemented in operational models are simple functions of the relative humidity, gained by fitting to measurement data, and can be evaluated analytically  \cite{gultepe2010probabilistic}.
The coefficients of the fit functions and even the function types, can vary vastly between different parametrizations, because they cannot be directly related to any physical process.
The dominating physical process that impacts visibility is Mie scattering, which is the scattering of plane electromagnetic waves on a sphere, for cases where the radius of the sphere and the wavelength of the incoming light have about the same size range \parencite{mie1908beitrage}. For sunlight different atmospheric particles satisfy this condition \parencite{wallace2006atmospheric}. Hence, the most accurate way of predicting visibility should be to resolve Mie scattering on hydrometeors and aerosols directly.
But because the spatial and temporal distribution of them is difficult to predict, there is a high level of overall uncertainty when trying to predict visibility by the use of Mie theory. As mentioned, probabilistic forecasts in general and ensemble prediction in particular can help to get meaningful results, even when the associated uncertainty is high.
Motivated by other attempts of probabilistic visibility forecasts \cite{chmielecki2011probabilistic, Roquelaure}, we combined ensemble methods and the theory of atmospheric scattering to a new scheme for probabilistic visibility prediction. 
This work presents and evaluates a new parametrization of visibility, based on Mie scattering theory, and how ensemble forecasting can be applied to predict visibility. It was carried out in cooperation with `Zentralanstalt für Meteorologie und Geodynamik' \cite{zamg}, the Austrian meteorological service.\\ \\
The necessary background theory to understand the basics of numerical weather prediction, the framework of this study, is provided in Chapter \ref{sec:bt}. In addition, the definition and mathematical derivation of visibility is provided and Mie theory is revised.\\
In Chapter \ref{sec:model} the model we used, is described, with special focus on the pattern generator for stochastic perturbations. The details of the new visibility parametrization and the benchmark  parametrizations are explained in Chapter \ref{sec:param}. Methods and set-up, including the evaluation strategies are the subject of Chapter \ref{sec:methods}. Then the results are presented and discussed in Chapter \ref{sec:results}. Finally, an outlook on potential future studies and concluding remarks, about strengths and weaknesses of the visibility prediction method is given in Chapter \ref{sec:con_n_out}.\\
\\
All used variables and abbreviations are listed in Appendix \ref{sec:tabelappendix} in Table \ref{tab: variables} and Table \ref{tab:abbrivitations}.



