!!!!!!!!!!!!!!!!!!!!!!INTERPRAETATION VON CLEMENS:
27.1.2017: 
Hochdruckgebiet mit Kern über Ungarn/Rumänien, Tiefdruckgebiete mit zugehörigen Frontensystemen über dem Ostatlantik. Österreich liegt im Osten noch im Einflussbereich des Hochs, während sich im Westen eine Kaltfront mit Wolken nähert. Schwacher Wind und großteils sonnige Verhältnisse. Lokale Nebelfelder im bayrischen Alpenvorland und im Nordosten Österreichs, bzw. in Tschechien. Temperaturen am Morgen tlw. Unter -10 Grad, im Nebel um -5C.

3.7.2016:
Höhentrog mit Kern über Norwegen. Hochdruck über Osteuropa. 
Kaltfront hat in der Nacht auf 3.7. Österreich überquert und atlantische Luftmassen zurückgelassen. Die starken Niederschlage in Verbindung mit der Kaltfront haben den Osten Österreichs am Vormittag des 3.7. verlassen. Letzte Schauer im Alpenraum. Auflockerungen gegen Mittag und am Nachmittag. Mit Sonneneinstrahlung aber wieder Quellwolkenbildung. Relativ kühl bei Temperaturen bis maximal 20 Grad.