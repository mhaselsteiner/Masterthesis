\begin{abstract}
    
    This  Master's thesis aims to describe and evaluate a new parametrization for forecasting visibility within numerical weather prediction. The parametrization represents Mie scattering on atmospheric aerosols and hydrometeors and is implemented in a numerical weather prediction model, written in Fortran.\\
    To examine the uncertainty in the parametrization and its effects, ensemble prediction is used. A new local perturbation scheme is introduced. For the verification of the visibility forecast, a logarithmic score  and the Ranked Probability Skill Score are chosen to evaluate the predicted values against the observed values, which were taken at meteorological observation sites on the domain of the forecast.\\
    The new parametrization in its current implementation outperforms the reference parametrization only in a few cases. However, we have reason to believe that further improvements the aerosol scheme will result in a significant increase of performance. The new perturbation scheme enhanced the ensemble skill remarkably in the whole verification period. The scheme can also be generalized and applied to other parametrizations or model parts of similar kind.
        
\end{abstract}